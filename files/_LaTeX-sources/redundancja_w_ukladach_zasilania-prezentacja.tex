\PassOptionsToPackage{unicode=true}{hyperref}
\documentclass[aspectratio=169]{beamer} % 1610, 149, 54, 43 and 32
%\documentclass[aspectratio=43]{beamer} % 1610, 149, 54, 43 and 32

\usepackage{fontspec}
\usepackage{fancyvrb, fvextra}
\usepackage{xcolor, graphicx}

% multilingual support
\usepackage{polyglossia}
\setdefaultlanguage{polish} % by default polish settings and names
                            % only this, NOT \RequirePackage{polski} NOR \RequirePackage[polish]{babel}
\setotherlanguages{english} % provide english language option too (for code printing)

\usepackage[ampersand]{easylist}

\usepackage{tikz}
\usetikzlibrary{positioning} % for positionig nodes with `right = of X`
\usetikzlibrary{arrows.meta, decorations.markings} % for arrows formating in tikzpicture
\usetikzlibrary{shapes} % for elipse nodes

\usetheme{CambridgeUS}
\usecolortheme{dolphin}
\setbeamertemplate{navigation symbols}{}

\title[Redundancja systemów zasilania]{Redundancja systemów zasilania – czy są rozwiązania lepsze od 2N?}
\author[Robert Paciorek]{Robert Paciorek\\ <rrp@opcode.eu.org>}
\date{Kraków 2019-10-24
%\\Konferencja ,,Data Center''
}

\begin{document}

\begin{frame}
\titlepage
\end{frame}

\section{Podstawy zasilania serwerowni}

\subsection{Zasilanie gwarantowane}

\begin{frame}[fragile]
\begin{easylist}[itemize]
& system zasilania gwarantowanego
&& czemu służy system zasilania gwarantowanego?
&& jakie są jego główne komponenty?
\end{easylist}
\end{frame}

\subsection{Redundancja}

\begin{frame}[fragile]
\begin{easylist}[itemize]
& czym jest redundancja?

& po co stosować redundancję?
&& zwiększenie niezawodności systemu, z dopuszczeniem planowych i nieplanowych wyłączeń
&& zapewnienie możliwości serwisowania bez wyłączeń
&& zapewnienie odporności na dowolną (pojedynczą) awarię
\end{easylist}
\end{frame}


\section{Typowe układy zasilania energetycznego serwerowni}

\subsection{podstawowy układ zasilania}

\begin{frame}
\begin{center}\begin{tikzpicture}[-, node distance=1.4em, semithick]
\tikzstyle{element}=[draw, minimum height=1.6em, minimum width=5em]

\node[element] (UTL_A) {Linia ,,miejska''};
\node[element] (SG_UTL_A) [below = of UTL_A] {Rozdzielnica dystrybucyjna};
\node[element] (SERW)  [below = of SG_UTL_A] {Zasilanie serwerów};

\draw (UTL_A) -- (SG_UTL_A);
\draw (SG_UTL_A) -- (SERW);
\end{tikzpicture}\end{center}
\end{frame}

\subsection{podstawowy układ zasilania}

\begin{frame}
\begin{center}\begin{tikzpicture}[-, node distance=1.4em, semithick]
\tikzstyle{element}=[draw, minimum height=1.6em, minimum width=5em]

\node[element] (UTL_A) {Linia ,,miejska''};
\node[element] (SG_UTL_A) [below = of UTL_A] {Rozdzielnica wejściowa};
\node[element] (UPS)  [below = of SG_UTL_A] {UPS};
\node[element] (SG_UPS_A) [below = of UPS]   {Rozdzielnica dystrybucyjna};
\node[element] (SERW)  [below = of SG_UPS_A] {Zasilanie serwerów};

\draw (UTL_A) -- (SG_UTL_A);
\draw (SG_UTL_A) -- (UPS);
\draw (UPS) -- (SG_UPS_A);
\draw (SG_UPS_A) -- (SERW);
\end{tikzpicture}\end{center}
\end{frame}

\subsection{redundancja źródeł}

\begin{frame}
\begin{center}\begin{tikzpicture}[-, node distance=1.4em, semithick]
\tikzstyle{element}=[draw, minimum height=1.6em, minimum width=5em]
\tikzstyle{invisible}=[inner sep=0, outer sep = 0pt, minimum height=1.6em, minimum width=0]

\node[element] (UTL_A) {Linia ,,miejska''};
\node[invisible] (UTL) [right = 0.8em of UTL_A] {};
\node[element] (UTL_B) [right = 0.8em of UTL] {???};
\node[element] (ATS_UTL) [below = of UTL] {Samoczynne Załączenie Rezerwy};
\node[element] (SG_UTL_A) [below = of ATS_UTL] {Rozdzielnica wejściowa};
\node[element] (UPS)  [below = of SG_UTL_A] {UPS};
\node[element] (SG_UPS_A) [below = of UPS]   {Rozdzielnica dystrybucyjna};
\node[element] (SERW)  [below = of SG_UPS_A] {Zasilanie serwerów};

\draw (UTL_A) |- (ATS_UTL.north);
\draw (UTL_B) |- (ATS_UTL.north);
\draw (ATS_UTL) -- (SG_UTL_A);
\draw (SG_UTL_A) -- (UPS);
\draw (UPS) -- (SG_UPS_A);
\draw (SG_UPS_A) -- (SERW);
\end{tikzpicture}\end{center}
\end{frame}

\begin{frame}[fragile]
\begin{easylist}[itemize]
& druga linia miejska czy agregat prądotwórczy?
&& ekonomicznie zależy od lokalnych uwarunkowań (m.in. prawdopodobieństwa co najmniej kilkuminutowego równoczesnego zaniku obu linii)
&& teoretycznie – agregat prądotwórczy
\end{easylist}
\end{frame}

\subsection{TIER I}

\begin{frame}
\begin{center}\begin{tikzpicture}[-, node distance=1.4em, semithick]
\tikzstyle{element}=[draw, minimum height=1.6em, minimum width=5em]
\tikzstyle{invisible}=[inner sep=0, outer sep = 0pt, minimum height=1.6em, minimum width=0]

\node[element] (UTL_A) {Linia ,,miejska''};
\node[invisible] (UTL) [right = 0.8em of UTL_A] {};
\node[element] (UTL_B) [right = 0.8em of UTL] {Generator};
\node[element] (SG_UTL_A) [below = of UTL] {Rozdzielnica wejściowa z SZR};
\node[element] (UPS)  [below = of SG_UTL_A] {UPS};
\node[element] (SG_UPS_A) [below = of UPS]   {Rozdzielnica dystrybucyjna};
\node[element] (SERW)  [below = of SG_UPS_A] {Zasilanie serwerów};

\draw (UTL_A) |- (UTL_A |- SG_UTL_A.north);
\draw (UTL_B) |- (UTL_B |- SG_UTL_A.north);
\draw (SG_UTL_A) -- (UPS);
\draw (UPS) -- (SG_UPS_A);
\draw (SG_UPS_A) -- (SERW);
\end{tikzpicture}\end{center}
\end{frame}


\subsection{Bypass}

\begin{frame}
\begin{center}\begin{tikzpicture}[-, node distance=1.4em, semithick]
\tikzstyle{element}=[draw, minimum height=1.6em, minimum width=5em]
\tikzstyle{invisible}=[inner sep=0, outer sep = 0pt, minimum height=1.6em, minimum width=0]

\node[element] (UTL_A) {Linia ,,miejska''};
\node[invisible] (UTL) [right = 0.8em of UTL_A] {};
\node[element] (UTL_B) [right = 0.8em of UTL] {Generator};
\node[element] (SG_UTL_A) [below = of UTL] {Rozdzielnica wejściowa z SZR};
\node[element] (UPS)  [below = of SG_UTL_A] {UPS};
\node[element] (SG_UPS_A) [below = of UPS]   {Rozdzielnica dystrybucyjna};
\node[element] (SERW)  [below = of SG_UPS_A] {Zasilanie serwerów};

\draw (UTL_A) |- (UTL_A |- SG_UTL_A.north);
\draw (UTL_B) |- (UTL_B |- SG_UTL_A.north);
\draw (SG_UTL_A) -- (UPS);
\draw (UPS) -- (SG_UPS_A);
\node[invisible, minimum height=0, right = 0.9em of SG_UTL_A] (UPSby) {};
\draw[dashed] (SG_UTL_A) -| (UPSby); \draw[dashed] (UPSby) |- (SG_UPS_A);
\draw (SG_UPS_A) -- (SERW);
\end{tikzpicture}\end{center}
\end{frame}

\begin{frame}[fragile]
\begin{easylist}[itemize]
& zewnętrzny bypass pozwala na serwisowanie UPSów
&& nie zapewniają one w trakcie tych prac pewności zasilania
&& w tym celu można skorzystać z agregatu prądotwórczego
\end{easylist}
\end{frame}

\subsection{TIER II}

% który z elementów w torze najbardziej zawodny? -> nadmiarowość UPS

\begin{frame}
\begin{center}\begin{tikzpicture}[-, node distance=1.4em, semithick]
\tikzstyle{element}=[draw, minimum height=1.6em, minimum width=5em]
\tikzstyle{invisible}=[inner sep=0, outer sep = 0pt, minimum height=1.6em, minimum width=0]

\node[element] (UTL_A) {Linia ,,miejska''};
\node[invisible] (UTL) [right = 0.8em of UTL_A] {};
\node[element] (UTL_B) [right = 0.8em of UTL] {Generator (N+1)};
\node[element] (SG_UTL_A) [below = of UTL] {Rozdzielnica wejściowa z SZR};
\node[element] (UPS)  [below = of SG_UTL_A] {UPS (N+1)};
\node[element] (SG_UPS_A) [below = of UPS]   {Rozdzielnica dystrybucyjna};
\node[element] (SERW)  [below = of SG_UPS_A] {Zasilanie serwerów};

\draw (UTL_A) |- (UTL_A |- SG_UTL_A.north);
\draw (UTL_B) |- (UTL_B |- SG_UTL_A.north);
\draw (SG_UTL_A) -- (UPS);
\draw (UPS) -- (SG_UPS_A);
\draw (SG_UPS_A) -- (SERW);
\end{tikzpicture}\end{center}
\end{frame}



\subsection{Zasilanie dwustronne}

\begin{frame}[fragile]
\begin{easylist}[itemize]
& umożliwienie serwisowania elementów systemu wymaga nadmiarowości
&& N+1 UPSów pozwala na wykonanie (bez przechodzenia na bypass) części (ale nie wszystkich) prac niemożliwych przy N UPSach
&& wiele prac w torze zasilania (oraz eliminacja skutków niektórych awarii) wymaga drugiego toru zasilania
& większość sprzętu IT pozwala na równoczesne zasilanie z dwóch źródeł
& zapewnia to odporność na awarię jednego z zasilaczy w urządzeniu albo jednego ze źródeł
\end{easylist}
\end{frame}

% serwery mają kilka zasilaczy -> strony -> czy strony z jednego źródła mają sens? -> serwisowo (rozbudowa rozdzielnicy, pomiary) TAK

% \begin{frame}
% \begin{center}\begin{tikzpicture}[-, node distance=1.4em, semithick]
% \tikzstyle{element}=[draw, minimum height=1.6em, minimum width=5em]
% \tikzstyle{invisible}=[inner sep=0, outer sep = 0pt, minimum height=1.6em, minimum width=0]
% 
% \node[element] (UTL_A) {Linia ,,miejska''};
% \node[invisible] (UTL) [right = 0.8em of UTL_A] {};
% \node[element] (UTL_B) [right = 0.8em of UTL] {Generator (N+1)};
% \node[element] (SG_UTL_A) [below = of UTL] {Rozdzielnica wejściowa z SZR};
% \node[element] (UPS)  [below = of SG_UTL_A] {UPS (N+1)};
% \node[invisible] (SG_UPS) [below = of UPS]   {};
% \node[element] (SG_UPS_A) [left = 0.7em of SG_UPS]   {Rozdzielnica dystrybucyjna ,,A''};
% \node[element] (SG_UPS_B) [right = 0.7em of SG_UPS]   {Rozdzielnica dystrybucyjna ,,B''};
% \node[element] (SERW)  [below = of SG_UPS] {Zasilanie serwerów};
% 
% \draw (UTL_A) |- (UTL_A |- SG_UTL_A.north);
% \draw (UTL_B) |- (UTL_B |- SG_UTL_A.north);
% \draw (SG_UTL_A) -- (UPS);
% \draw (UPS) -- (SG_UPS.center);
% \draw[dashed] (SG_UTL_A) -| (SG_UPS_A);
% \draw[dashed] (SG_UTL_A) -| (SG_UPS_B);
% \draw (SG_UPS.center) -- (SG_UPS_A);
% \draw (SG_UPS.center) -- (SG_UPS_B);
% \draw (SG_UPS_A) |- (SERW);
% \draw (SG_UPS_B) |- (SERW);
% \end{tikzpicture}\end{center}
% \end{frame}

\subsection{TIER III}

% prawdziwe strony zasilania -> powielamy układ, ale czy w całości?
% 	-> współdzielony agregat
% 	-> drugi tor bez UPSa (czy ma sens? -> uodparniamy się na awarię UPSa, umożliwiamy pełniejszy serwis niż przy pomocy bypassu)
% 	-> pełny drugi tor => 2N

\begin{frame}
\begin{center}\begin{tikzpicture}[-, node distance=1.4em, semithick]
\tikzstyle{element}=[draw, minimum height=1.6em, minimum width=5em]
\tikzstyle{invisible}=[inner sep=0, outer sep = 0pt, minimum height=1.6em, minimum width=0]

\node[element] (UTL) {Linia ,,miejska''};
\node[element] (GEN) [below = of UTL] {Generator (N+1)};

\node[invisible] (inv4) [left = 4em of GEN]   {};
\node[invisible] (inv3) [right = 4em of GEN]   {};

\node[element,align=center] (SG_UTL_A) [below = of inv4] {Rozdzielnica\\ wejściowa ,,A'' z SZR};
\node[element,align=center] (SG_UTL_B) [below = of inv3] {Rozdzielnica\\ wejściowa ,,B'' z SZR};

\node[element] (UPS_A)  [below = of SG_UTL_A] {UPS (N+1)};
\node[element] (SG_UPS_A) [below = of UPS_A]   {Rozdzielnica dystrybucyjna ,,A''};
\node[element] (SG_UPS_B) [] at (SG_UTL_B |- SG_UPS_A)   {Rozdzielnica dystrybucyjna ,,B''};
\node[invisible] (inv1) [below = of SG_UPS_A]   {};
\node[element] (SERW)  [] at (inv1 -| UTL) {Zasilanie serwerów};

\draw (UTL) -| (SG_UTL_A);
\draw (UTL) -| (SG_UTL_B);
\draw (GEN.-110) |- (SG_UTL_A);
\draw (GEN.-70)  |- (SG_UTL_B);

\draw (SG_UTL_A) -- (UPS_A);
\draw (UPS_A) -- (SG_UPS_A);
\draw (SG_UTL_B) -- (SG_UPS_B);
\draw (SG_UPS_A) |- (SERW);
\draw (SG_UPS_B) |- (SERW);
\end{tikzpicture}\end{center}
\end{frame}

\subsection{TIER IV}


\begin{frame}
\begin{center}\begin{tikzpicture}[-, node distance=1.4em, semithick]
\tikzstyle{element}=[draw, minimum height=1.6em, minimum width=5em]
\tikzstyle{invisible}=[inner sep=0, outer sep = 0pt, minimum height=1.6em, minimum width=0]

\node[element] (GEN_A){Generator ,,A''};
\node[element] (UTL) [right = of GEN_A] {Linia ,,miejska''};
\node[element] (GEN_B) [right = of UTL] {Generator ,,B''};

\node[element,align=center] (SG_UTL_A) [below = of GEN_A] {Rozdzielnica\\ wejściowa ,,A'' z SZR};
\node[element,align=center] (SG_UTL_B) [below = of GEN_B] {Rozdzielnica\\ wejściowa ,,B'' z SZR};

\node[element] (UPS_A)  [below = of SG_UTL_A] {UPS ,,A''};
\node[element] (UPS_B)  [below = of SG_UTL_B] {UPS ,,B''};
\node[element] (SG_UPS_A) [below = of UPS_A]   {Rozdzielnica dystrybucyjna ,,A''};
\node[element] (SG_UPS_B) [below = of UPS_B]   {Rozdzielnica dystrybucyjna ,,B''};
\node[invisible] (inv1) [below = of SG_UPS_A]   {};
\node[element] (SERW)  [] at (inv1 -| UTL) {Zasilanie serwerów};


% \draw (UTL_A) |- (UTL_A |- SG_UTL_A.north);
% \draw (UTL_B) |- (UTL_B |- SG_UTL_A.north);
\draw (GEN_A) -- (SG_UTL_A);
\draw (GEN_B) -- (SG_UTL_B);
\draw (UTL.-110) |- (SG_UTL_A);
\draw (UTL.-70)  |- (SG_UTL_B);

\draw (SG_UTL_A) -- (UPS_A);
\draw (UPS_A) -- (SG_UPS_A);
\draw (SG_UTL_B) -- (UPS_B);
\draw (UPS_B) -- (SG_UPS_B);
\draw (SG_UPS_A) |- (SERW);
\draw (SG_UPS_B) |- (SERW);
% \draw (SG_UPS_B) |- (SERW);
\end{tikzpicture}\end{center}
\end{frame}

\begin{frame}[fragile]
\begin{center}
\LARGE Pełna separacja torów.\\ Żadna awaria w torze ,,A'' nie powinna wpłynąć na działanie toru ,,B'' i vice versa.
\end{center}

\vspace{1cm}

\begin{easylist}[itemize]
& separacja fizyczna pomieszczeń
& wydzielenia pożarowe
& osobne przyciski PWP
& ...
\end{easylist}
\end{frame}


\section{Jak robić lepiej?}

\subsection{}

\begin{frame}[fragile]
\begin{easylist}[itemize]
& czy warto coś dodać do tego schematu?
&& zewnętrzny bypass UPSów?
&& sprzęgło pomiędzy rozdzielnicą wejściową ,,A'' i ,,B''?
&& sprzęgło pomiędzy rozdzielnicą dystrybucyjna ,,A'' i ,,B''?
&& STSy pomiędzy stronami ,,A'' i ,,B'' pozwalające na zasilenie obu stron serwera z jednego toru zasilania?
&& N+1 na elementach w torze?
&& ...?
& jak to robić?
&& sprzęgła muszą mieć podwójne wyłączniki – osobne od strony toru ,,A'' i osobne od strony toru ,,B''
&& STSy muszą być dwa – osobny tworzący „A'” oraz osobny dla „B'”, a każdy serwer zasilany dwustronnie (z „A'” i „B'”)
\end{easylist}
\end{frame}

% 2 ścieżki + 2x STS - czy to coś zmienia => zależy, np. od czasu naprawy uszkodzonego zasilacza

\begin{frame}
\begin{center}
\LARGE Każdy element stanowi potencjalne źródło problemów i awarii – jego wprowadzenie do układu powinno być uzasadnione.
\end{center}
\end{frame}

% nawiązanie do wprowadzenia UPSa (może byc przyczyną awarii, ale zabezpiecza przed bardziej prawdopodobną awarią sieci)


\subsection{Reguła KISS}

\begin{frame}[fragile]
\begin{easylist}[itemize]
& unikajmy komplikowania, proste rozwiązania zapewniają:
&& łatwiejsze użytkowanie i utrzymanie – mniej ,,wiedzy tajemnej''
&& mniejsze ryzyko błędów ludzkich (zwłaszcza w trakcie ,,zwalczania'' awarii)
&& (często) mniejsze koszty inwestycyjne i utrzymaniowe
& prostota wymaga konsekwencji – nie robimy wyjątków
\end{easylist}
\end{frame}


\subsection{Redundancja systemów}

\begin{frame}
\begin{center}\begin{tikzpicture}[-, node distance=1.4em, semithick]
\tikzstyle{element}=[draw, minimum height=3em, align=center]
\tikzstyle{invisible}=[inner sep=0, outer sep = 0pt, minimum height=0, minimum width=0]
\node[element]   (PWR_A) {\textbf{,,A''}\\3x100kW {\color{blue} x2}};
\node[invisible, minimum height=3em, minimum width=3em] (PWR) [right = of PWR_A] {};
\node[element]   (PWR_B) [right = of PWR]  {\textbf{,,B''}\\3x100kW {\color{blue} x2}};

\node[element] (SER_1) [below = of PWR] {\textbf{1}\\200kW = 2x100kW};
\node[element] (SER_2)  [below = of SER_1] {\textbf{2}\\200kW = 2x100kW};
\node[element] (SER_3)  [below = of SER_2] {\textbf{3}\\200kW = 2x100kW};

\draw (PWR_A) |- (SER_1);
\draw (PWR_A) |- (SER_2);
\draw (PWR_A) |- (SER_3);

\draw (PWR_B) |- (SER_1);
\draw (PWR_B) |- (SER_2);
\draw (PWR_B) |- (SER_3);
\end{tikzpicture}\end{center}
łączna moc zainstalowana systemów zasilających = 2x3x200kW = 1200kW
łączna moc dostępna systemów zasilających = 3x200kW = 600kW
\end{frame}

\subsection{,,Mieszanie stron''}

\begin{frame}
\begin{center}\begin{tikzpicture}[-, node distance=1.4em, semithick]
\tikzstyle{element}=[draw, minimum height=3em, align=center]
\tikzstyle{invisible}=[inner sep=0, outer sep = 0pt, minimum height=0, minimum width=0]
\node[element]   (PWR_A) {\textbf{,,A''}\\2x100kW {\color{blue} +100kW}};
\node[element]   (PWR_B) [right = 2.4em of PWR_A]  {\textbf{,,B''}\\2x100kW {\color{blue} +100kW}};
\node[element]   (PWR_C) [right = 2.4em of PWR_B]  {\textbf{,,C''}\\2x100kW {\color{blue} +100kW}};

\node[element] (SER_1) [below = of PWR_A.south east] {\textbf{1}\\200kW = 2x100kW};
\node[element] (SER_2)  [below = of PWR_C.south west] {\textbf{2}\\200kW = 2x100kW};
\node[element] (SER_3)  [below = of PWR_B |- SER_1.south] {\textbf{3}\\200kW = 2x100kW};

\draw (PWR_A.-140) |- (SER_1);
\draw (PWR_A.-140) |- (SER_3);

\draw (PWR_B) |- (SER_1);
\draw (PWR_B) |- (SER_2);

\draw (PWR_C.-50) |- (SER_2);
\draw (PWR_C.-50) |- (SER_3);
\end{tikzpicture}\end{center}
łączna moc zainstalowana systemów zasilających = 3x300kW = 900kW
łączna moc dostępna systemów zasilających = 3x200kW = 600kW
\end{frame}

\begin{frame}[fragile]
\begin{columns}\begin{column}{0.5\textwidth}
\textbf{Zalety}
\begin{easylist}[itemize]
& 2N w cenie N+1
& prostota koncepcyjna rozwiązania
& powielalność ,,modułu zasilającego''
\end{easylist}
\end{column}\begin{column}{0.5\textwidth}
\textbf{Wady}
\begin{easylist}[itemize]
& trudniejszy balans mocy
& bardziej złożony układ dystrybucyjny
& ciężko skalowalne dla większej ilości stron niż 5
& raczej dla dużych serwerowni
\end{easylist}
\end{column}\end{columns}
\end{frame}


\subsection{N+2}

\begin{frame}
\begin{center}\begin{tikzpicture}[-, node distance=1.4em, semithick]
\tikzstyle{element}=[draw, minimum height=3em, align=center]
\tikzstyle{invisible}=[inner sep=0, outer sep = 0pt, minimum height=0, minimum width=0]

\node[element] (SER_1) {\textbf{1}\\200kW =\\2x50{\color{blue}+2x50}};
\node[element] (A1) [above left = of SER_1.north west] {\textbf{A}\\50kW {\color{blue}+50kW}};
\node[element] (B1) [below left = of SER_1.south west] {\textbf{B}\\50kW {\color{blue}+50kW}};
\node[element] (C1) [above right = of SER_1.north east] {\textbf{C}\\50kW {\color{blue}+50kW}};
\node[element] (D1) [below right = of SER_1.south east] {\textbf{D}\\50kW {\color{blue}+50kW}};

\draw (A1) |- (SER_1.160);
\draw (B1) |- (SER_1.200);
\draw (C1) |- (SER_1.20);
\draw (D1) |- (SER_1.-20);
\end{tikzpicture}\end{center}

łączna moc zainstalowana systemów zasilających = 4x100kW = 400kW
łączna moc dostępna systemów zasilających = 4x50kW = 200kW
\textbf{odporność na awarię dwóch torów}
\end{frame}

\begin{frame}
Tutaj też można ,,pomieszać'' strony zasilania różnych komór.

\begin{center}\begin{tabular}{ c || c | c | c | c | c }
 	&	A	&	B	&	C	&	D	&	E \\ \hline
1	&	x	&	x	&	x	&	x	&	  \\
2	&	x	&	x	&	x	&		&	x \\
3	&	x	&	x	&		&	x	&	x \\
4	&	x	&		&	x	&	x	&	x \\
5	&		&	x	&	x	&	x	&	x \\
\end{tabular}\end{center}

Gdy wypadną \textbf{dwa} dowolne moduły obciążenie każdego z pozostałych wzrośnie o 75\%. Zatem każdy moduł musi być przewymiarowany o 75\% a nie 100\% jak w wcześniejszym przypadku.
\end{frame}

\begin{frame}[fragile]
\textbf{Zalety}
\begin{easylist}[itemize]
& odporność na dwie równoczesne dowolne awarie
& odporność na awarię w trakcie prac serwisowych
&& przy założeniu nie serwisowania więcej niż jednego toru na raz
& realizowalność na istniejącym sprzęcie
&& serwery 4 zasilaczowe najczęściej do pracy wymagają zasilenia dwóch dowolnych zasilaczy
&& redundantny stack switchy dwuzasilaczowych działa przy zasileniu dowolnego jednego zasilacza
& kompatybilność z sprzętem dwuzasilaczowym
&& klaster HA dwóch urządzeń dwuzasilaczowych jest odporny na awarię 3 torów zasilania
&& korzystając jedynie z dwóch stron tracimy tylko dodatkowe atuty tego rozwiązania, ale zachowujemy maksymalną niezawodność zasilania takiego urządzenia
% & możliwość dywersyfikacji technologii
% && eliminacja takich samych awarii na różnych stronach
% && problem serwisowy / utrzymaniowy
\end{easylist}
\end{frame}

\begin{frame}[fragile]
\textbf{Wady}
\begin{easylist}[itemize]
& złożony układ dystrybucyjny – 4 tory zasilania do odbiorników
& większa nadmiarowość urządzeń niż w omawianym układzie ,,mieszania stron''
& brak bezpośrednich korzyści dla urządzeń 2 zasilaczowych, chyba że:
&& użyjemy ATSów lub STSów
&& złożymy dwa takie urządzenia w klaster HA
\end{easylist}
\end{frame}


\section{Pytania?}

\begin{frame}
\begin{center}
\LARGE Pytania?
\end{center}
\end{frame}


\section{Licencja}

\begin{frame}
Copyright © 2019, Robert Ryszard Paciorek <rrp@opcode.eu.org>

\footnotesize\vspace{1.5em}
To jest wolny i otwarty dokument. Redystrybucja, użytkowanie i/lub modyfikacja SĄ DOZWOLONE na warunkach licencji MIT.

\vspace{1.5em}
This is free and open document. Redistribution, use and/or modify ARE PERMITTED under the terms of the MIT license.

\tiny\vspace{2em}
Permission is hereby granted, free of charge, to any person obtaining a copy
of this software and associated documentation files (the "Software"), to deal
in the Software without restriction, including without limitation the rights
to use, copy, modify, merge, publish, distribute, sublicense, and/or sell
copies of the Software, and to permit persons to whom the Software is
furnished to do so, subject to the following conditions:

\vspace{1em}
The above copyright notice and this permission notice shall be included in all
copies or substantial portions of the Software.

\vspace{1em}
THE SOFTWARE IS PROVIDED "AS IS", WITHOUT WARRANTY OF ANY KIND, EXPRESS OR
IMPLIED, INCLUDING BUT NOT LIMITED TO THE WARRANTIES OF MERCHANTABILITY,
FITNESS FOR A PARTICULAR PURPOSE AND NONINFRINGEMENT. IN NO EVENT SHALL THE
AUTHORS OR COPYRIGHT HOLDERS BE LIABLE FOR ANY CLAIM, DAMAGES OR OTHER
LIABILITY, WHETHER IN AN ACTION OF CONTRACT, TORT OR OTHERWISE, ARISING FROM,
OUT OF OR IN CONNECTION WITH THE SOFTWARE OR THE USE OR OTHER DEALINGS IN THE
SOFTWARE.

\end{frame}

\end{document}
